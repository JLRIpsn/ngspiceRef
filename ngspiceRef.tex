%An ngspice reference sheet
%By JLRIpsn

\documentclass[a4paper]{article}

%need to include font that supports < and > symbols
\usepackage[T1]{fontenc}

\author{JLRIpsn}
\title{An ngspice Quick Reference}

\begin{document}

\maketitle
\section{File Layout Summary}

*Ngspice title and/or description (firstline) \\
.include <directory/file>\\
\\
.param paramVal=10kV\\
*Circuit layout\\
VName 0 node2 DC voltageForDCanalysis SIN(DCopVoltage amplitudeVoltage freq)\\
\\
R1 0 node2 10k\\
C1 0 node1 10uF\\
L1 node1 node2 5mH\\

*tests\\
.control\\
tran <timeForTransientSim> <simStepSize>\\
meas tran valName PP v(node1)\\
*PP= peak to peak\\
echo "words"\\
plot v(node1)\\
.endc\\
\\
.end
\section{Basic Circuit Components}
\begin{table}[!ht]
	\begin{tabular}{| c | c |}
		\hline
		DC Vsource & V<name> <node-> <node+> DC <value(V)> \\
		\hline
		AC Vsource & V<name> <node-> <node+> DC 0V SIN(<DCoffset(V)> <amplitude(V)> <freq(Hz)> \\
		\hline
		Resistor & R<name> <node1> <node2> \\
		\hline
		Capacitor & C<name> <node1> <node2> <value(F)> \\
		\hline
		Inductor & L<name> <node1> <node2> <value(H)> \\
		\hline
		Diode & D<name> ... \\
		\hline
		MOSFET & M<name> ... \\
		\hline
		BJT & ... \\
		\hline
		Op-Amp &  ... \\
		\hline
	\end{tabular}
	\caption{\label{tab:comps} Basic circuit components}
\end{table}

\newpage
\section{Test and measurement basics}
\subsection{DC}
\subsection{Transient}
\subsection{AC}

\end{document}
